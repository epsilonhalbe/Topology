\documentclass[a4paper,12pt]{article}
\usepackage[german]{babel}
\usepackage{amsfonts,amsmath,amssymb,bbm,wasysym,mathrsfs}
\usepackage{enumerate}
\usepackage{makeidx}
\usepackage[german,norefeq,norefpage]{nomencl}
\makeindex
\usepackage[xindy]{glossaries}
\makeglossary
\usepackage{graphicx,color}
\usepackage[nottoc]{tocbibind}
\usepackage[only, lightning]{stmaryrd}
%\usepackage[squaren]{SIunits}
\makeatletter
\providecommand{\LyX}{L\kern-.1667em\lower.25em\hbox{Y}\kern-.125emX\@}
\providecommand{\fertig}{$\blacksquare$}
\usepackage{geometry}
\usepackage{multicol}
\usepackage{fancyhdr}
\pagestyle{fancy}
\hoffset=5 pt
\headheight =28pt
\headsep=20pt
\footskip = 40pt
\textheight =650pt
\fancyfoot{\empty}
\fancyfoot[L]{\today}
\fancyfoot[R]{\thepage}
\fancyhead{\empty}
\lhead{
\begin{picture}(0,0)
\put(-80,-340){\line(10,0){10}}
\end{picture}
}
\rhead{\hfill {\bf \leftmark}\newline\rightmark}
\usepackage{tocloft}
\addtolength{\cftsecindent}{-1em}
\addtolength{\cftsecnumwidth}{1em}
\usepackage{xy}
	\input xy
	\xyoption{all}
\usepackage{theorem}
	\theorembodyfont{\upshape}
	\newtheorem{definition}{Definition:}[section]
	\newtheorem{satz}[definition]{Satz:}
	\newtheorem{kor}[definition]{Korollar:}
	\newtheorem{prop}[definition]{Proposition}
	\newtheorem{lemma}[definition]{Lemma:}
	\newtheorem{bem}[definition]{Bemerkung:}
	\newtheorem{beispiel}[definition]{Beispiel:}
	\newtheorem{beob}[definition]{Beobachtung:}

	\renewcommand{\thesection}{\Roman{section}}
	\renewcommand{\thesubsection}{\arabic{subsection}}
	
	\newcommand{\beweis}[1]{\begin{description}\item[\ul{Beweis:}] #1 \hfill\fertig\end{description}}
	
	\newcommand{\N}{\mathbb N} \newcommand{\Z}{\mathbb Z} \newcommand{\R}{\mathbb R}
	\newcommand{\Q}{\mathbb Q}\newcommand{\C}{\mathbb C}
	\newcommand{\surj}{\twoheadrightarrow}
	\newcommand{\inj}{\rightarrowtail}
	\newcommand{\ra}{\rightarrow}
	\newcommand{\ramit}[1]{\stackrel{#1}{\longrightarrow}}
	\newcommand{\impl}{\Rightarrow}\newcommand{\lpmi}{\Leftarrow}
	\newcommand{\implmit}[1]{\stackrel{#1}{\Longrightarrow}}
	\newcommand{\lpmimit}[1]{\stackrel{#1}{\Longleftarrow}}
	\newcommand{\gleichmit}[1]{\stackrel{#1}{=}}
	\newcommand{\seq}{\subseteq }
	\newcommand{\ul}{\underline }
	\newcommand{\inv}[1]{#1^{-1}}
	\newcommand{\eq}{\Leftrightarrow}
	\newcommand{\eqmit}[1]{\stackrel{#1}{\eq}}
	\newcommand{\eps}{\varepsilon}
	\renewcommand{\ker}[1]{{\bf ker} #1 }
	\newcommand{\norm}[1]{\|{ #1 }\|}
	\newcommand{\ol}[1]{\overline{#1}}
	\newcommand{\durch}[1]{\frac{1}{#1}}
	\newcommand{\leer}{\emptyset}
	\newcommand{\et}{\wedge}
	\newcommand{\vel}{\vee}
	\newcommand{\ovec}{\mathfrak o}
	\newcommand{\innprod}[2]{<\hspace*{-0.15cm} #1|#2\hspace*{-0.15cm} >}
	\newcommand{\wid}{\mbox{\large\mbox{$\lightning$}}}
	\newcommand{\todo}[1]{\mbox{\textcolor{red}{$\clubsuit$} #1 \textcolor{red}{$\spadesuit$}}}
	\newcommand{\pr}{\mbox{pr}}
	\newcommand{\blitzmit}[2]{\stackrel{\stackrel{#2}{\lightning}}{#1}}
